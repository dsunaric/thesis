

%Every organization has established internal processes that have to be managed. In order to be able to analyse and optimize these processes they firstly need to be specified. The established standard for visualizing business processes graphically is \gls{bpmn}. \cite{fundamentals} 

%Furthermore a graphical \gls{bpmn} model can build a bridge between \gls{bpm} experts and software engineers when there should be a common understanding about how the process works. IT-specialists have an important role in \gls{bpm} since the goal should be, once the processes of an organization are defined, to automate them as much as possible using computers. \cite{praxishandbuch}

%The advantage of using \gls{bpmn} instead of other specification languages for business processes is that it can easily be automated using \gls{bpms}. These \gls{bpms} can directly execute a \cite{executable-bpmn}. A \gls{bpmn} model that is usually used by \gls{bpm} experts, is called a \gls{business-oriented-bpmn} and has to be transformed into an \gls{executable-bpmn} in order to be automated. \cite{fundamentals} 


%\section{Motivation}
%Automated processes in large organizations are usually more complex than the \hyperref{fig:comparison-executable-conceptual}{Pizza ordering process} and consist of many steps 
%A shorter (and therefore simpler) model is easier to understand by both party's.

%Another reason for reducing the number of tasks in a process is the overhead for creating and persisting a task by a BPMN engine. This might be negligent for automated steps in BPMN but the overhead starts becoming notable when having a lot of user-tasks (non-automated tasks) since there is also the overhead of assigning tasks to users and communication between users.

%Another practical issue with having many nodes in BPMN models is the enterprise pricing of BPMN Engines. Some BPMN engines have a pricing scheme according to process instances passing a node so reducing the number of nodes without taking away expressiveness might have an impact on the pricing model and licence needed. 


%\section{Goal}

%\section{Structure of this thesis}
%The following section will give a brief introduction about the structure of this thesis. 


%\section{Evaluation approach}