\label{intro}Today \gls{wfms}s like \gls{camunda}  and \gls{alfresco} are widely used by big organizations \cite{camunda-customers} \cite{activity-customers}. Real-life processes being executed in \gls{wfms}s can be complex and have to be changed and reevaluated together with the software that is executed in these processes as requirements and guidelines for the specific domain change continuous. 

Using an \gls{wfms} to directly automate processes defined as \gls{bpmn} has the advantage to be readable for IT experts as well as domain experts. As a consequence both parties have a common understanding of the process and communication is easier when changes have to be made to the process or the software. For the cooperation between IT and business to work smoothly, \gls{executable-bpmn} should be as simple to understand as possible and be in line with known standards and best practices. 

Applying guidelines for good \gls{executable-bpmn}s does not only increase readability but can also have a direct impact on process costs. Due to the pricing scheme of some enterprise \gls{wfms}s, which take into account the total number of nodes passed by process instances, eliminating nonnecessary handovers and therefore reducing the number of nodes in the \gls{bpmn} model, can have an impact on the required license. 

However, all suggestions for implementing efficient process models can only be seen as a guideline since there are many reasons why a more complex process model is chosen over one that complies with most or all of the guidelines. Verification if the process is in compliance with legal requirements or benchmarking individual tasks could be reasons for deviating from guidelines.

\section{Goal of This Thesis}
The goal of this thesis is to give practical guidance on how to apply best practices and methodologies to create good \gls{executable-bpmn}s and to improve existing models to satisfy these best practices. 


\section{Structure of This Thesis}
This goal will be achieved by stating the current research on creating \gls{executable-bpmn}s in chapter \ref{chapter-2}. 

After that, this work will take a look at methods for transforming and improving existing \gls{executable-bpmn}s and how to use them to create models in line with the stated best practices. To compare two process models, this work will also provide methods for measuring metrics like time and cost of processes defined as \gls{bpmn} models. This will be shown in chapter \ref{chapter-3}

Along with this thesis, a software was developed that scans a given BPMN model for used Anti-patterns. The documentation and implementation details of the developed software can be found in \ref{chapter-4}.  

Finally, this thesis will provide detailed information about how to use mentioned guidelines together with the developed software in practice with a case study in chapter \ref{chapter-5}.