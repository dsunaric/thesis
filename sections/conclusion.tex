This thesis aimed to summarize best practices for creating and transforming executable BPMN models. Based on the best practices given in chapter \ref{chapter-2} and \ref{chapter-3}, the following suggestions can be implemented into a given BPMN process model:
\begin{enumerate}
	\item Comply with naming conventions
	\item Eliminate manual tasks
	\item Extend automation boundaries
	\item Complete the process model
	\item Merge consecutive tasks handled by the same resource
	\item Replace combinations of parallel and exclusive gateways with inclusive gateways
	\item Perform value added analysis and waste elimination
\end{enumerate}

The software described in chapter \ref{chapter-4} can be used to scan an \textit{.bpmn} file for violations of the mentioned best practices. The software then returns a list of BPMN elements that violate the given rule. This software is freely available as an open source project on GitHub:  \url{https://github.com/dsunaric/epms-service}. 

This thesis goes then on to demonstrate in a case study how the software can be used as an aid to implement the listed suggestions in chapter \ref{chapter-5}. 

Implementing any of the suggestions mentioned in this thesis into every process has to be evaluated. Purposefully violating best practices can have various reasons as mentioned in the introduction \ref{intro}, like compiling with legal guidelines or utilizing the benchmarking capabilities of the used \gls{wfms} to get an insight on the performance of individual activities. 

However, all suggestions for implementing efficient process models can only be seen as a guideline since there are many reasons why a more complex process model is chosen over one that complies with most or all of the guidelines. Verification if the process is in compliance with legal requirements could be one of those reasons.

While this thesis mentioned a few anti-patterns that are commonly used in executable BPMNS, there are probably many others left that could easily be automatically identified. Further research is needed to quantifiably determine the effectiveness of implementing the given suggestions with measures like the flow-node count or execution time. 